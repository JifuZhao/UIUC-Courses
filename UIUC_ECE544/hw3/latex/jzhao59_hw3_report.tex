\documentclass[11pt,letterpaper]{article}

\newenvironment{proof}{\noindent{\bf Proof:}}{\qed\bigskip}

\newtheorem{theorem}{Theorem}
\newtheorem{corollary}{Corollary}
\newtheorem{lemma}{Lemma} 
\newtheorem{claim}{Claim}
\newtheorem{fact}{Fact}
\newtheorem{definition}{Definition}
\newtheorem{assumption}{Assumption}
\newtheorem{observation}{Observation}
\newtheorem{example}{Example}
\newcommand{\qed}{\rule{7pt}{7pt}}

\newcommand{\solution}[4]{
\thispagestyle{plain} 
\newpage
\setcounter{page}{1}
\noindent
\begin{center}
\framebox{ \vbox{
\vspace{4mm}
\vspace{0.2in} 
{\centering \large\mbox{#3}}\\
\vspace{0.1in}
{#1 \hfill {Date: #2}}
}}
\end{center}
\markright{#1}
}

\newenvironment{algorithm}
{\begin{center}
\begin{tabular}{|l|}
\hline
\begin{minipage}{1in}
\begin{tabbing}
\quad\=\qquad\=\qquad\=\qquad\=\qquad\=\qquad\=\qquad\=\kill}
{\end{tabbing}
\end{minipage} \\
\hline
\end{tabular}
\end{center}}

\def\Comment#1{\textsf{\textsl{$\langle\!\langle$#1\/$\rangle\!\rangle$}}}



\usepackage{graphicx, amssymb, amsmath, listings, float, mathtools}
\usepackage{color, url}
\lstset{language = Python}
\lstset{breaklines}
\lstset{extendedchars=false}

\oddsidemargin 0in
\evensidemargin 0in
\textwidth 6.5in
\topmargin -0.6in
\textheight 9.0in

\begin{document}

\solution{\large Jifu Zhao}{\large 10/20/2016}{\bf \Large ECE 544NA \hspace{0.5cm} 
		Fall 2016 \hspace{0.5cm} Assignment 3}

\section*{\Large I. Pencil-and-Paper}
\begin{description}
%%%%%%%%%%%%%%%%%%%%%%%%%%%%%%%%%%%%%%%%%%%%%%%%%%%%%%%%%%%%%%%%%%%%%%%%%%%%%%%%%%%%%%%%%
% Problem 1
\item{\bf \large 1. E-Step}

Since that $p_{X|\Theta(x_i|\theta)}$ is described as:
\begin{equation}
	p_{X|\Theta(x_i|\theta)} = \sum_{h=1}^{m} w_h p_{V|H, \Theta}(x_i, h, \theta)
\end{equation}

Now, suppose the hidden variable is $y$, where $y$ could be 1, 2, $\cdots$, m. Then, we can calculate the expectation of the log-likelihood $E[log\mathcal{L}(\Theta)]$ as:
\begin{equation}
\begin{split}
E[log\mathcal{L}(\Theta)] = \sum_{k=1}^m [\sum_{i=1}^n log(\sum_{h=1}^m w_h p_{V|H, \Theta}(x_i, h, \theta | y=k))] \cdot \gamma_i(k)
\end{split} 
\end{equation}
where $\gamma_i(k)$ is the posterior probability for $y = k$, and $\gamma_i(k)$ is defined as:
\begin{equation}
	\gamma_i(k) = \frac{w_k \mathcal{N}(x_i; \mu_k, \Sigma_k)}{\sum_l w_l \mathcal{N}(x_i; \mu_l, \Sigma_l)}
\end{equation}

Also, notice that 
\begin{equation}
p_{V|H, \Theta}(x_i, h, \theta | y=k)) = 
	\begin{cases}
		\mathcal{N}(x_i; \mu_k, \Sigma_k)  & \quad \text{if } h = k \\
		0 & \quad \text{if } h \neq k
	\end{cases}
\end{equation}

So, 
\begin{equation}
\begin{split}
E[log\mathcal{L}(\Theta)] & = \sum_{i=1}^n \sum_{k=1}^m log(w_k p_{V|H, \Theta}(x_i, h=k, \theta)) \cdot \gamma_i(k) \\
						  & =  \sum_{i=1}^n \sum_{k=1}^m log(w_k \mathcal{N}(x_i; \mu_k, \Sigma_k))\cdot \frac{w_k \mathcal{N}(x_i; \mu_k, \Sigma_k)}{\sum_l w_l \mathcal{N}(x_i; \mu_l, \Sigma_l)} \\
						  & = \sum_{i=1}^n \sum_{k=1}^m log(w_k \mathcal{N}(x_i; \mu_k, \Sigma_k))\cdot \gamma_i(k)
\end{split} 
\end{equation}

So, in E-Step, the most important thing is to calculate $\gamma_i(k)$ for each i and k, where 
\begin{equation}
\gamma_i(k) = \frac{w_k \mathcal{N}(x_i; \mu_k, \Sigma_k)}{\sum_l w_l \mathcal{N}(x_i; \mu_l, \Sigma_l)}
\end{equation}


%%%%%%%%%%%%%%%%%%%%%%%%%%%%%%%%%%%%%%%%%%%%%%%%%%%%%%%%%%%%%%%%%%%%%%%%%%%%%%%%%%%%%%%%%
% Problem 2
\item{\bf \large 2. M-Step}

From above equation (5), our goal is the maximize the expectation of log-likelihood $E[log\mathcal{L}(\Theta)]$.

First, consider $w_k$. With the constraint of $\sum_k w_k = 1$, we have:
\begin{equation}
\frac{\partial}{\partial w_k} [\sum_{i=1}^n \sum_{k=1}^m log(w_k \mathcal{N}(x_i; \mu_k, \Sigma_k))\cdot \gamma_i(k) + \lambda(\sum_k w_k - 1)] = 0
\end{equation}

So, we have: 
\begin{equation}
\sum_{i=1}^n \frac{\gamma_i(k)}{w_k} + \lambda = 0
\end{equation}

Summing it over k from 1 to m and with the equation that $\sum_k \gamma_i(k) = 1$, we can have:
\begin{equation}
\lambda = n
\end{equation}

So, we have:
\begin{equation}
	w_k^{new} = \frac{1}{n} \sum_{i=1}^n \gamma_i(k)
\end{equation}

Now, let's consider $\mu_k$ and $\Sigma_k$. Following the steps in {\it A gentle tutorial of the EM algorithm and its application to parameter estimation for Gaussian mixture and hidden Markov models}, we can simply get the following result:
\begin{equation}
	\mu_k^{new} = \frac{\sum_{i=1}^n x_i \gamma_i(k)}{\sum_{i=1}^n \gamma_i(k)}
\end{equation}


\begin{equation}
	\Sigma_k^{new} = \frac{\sum_{i=1}^n (x_i - \mu_k^{new}) \cdot (x_i - \mu_k^{new})^T \cdot \gamma_i(k)}{\sum_{i=1}^n \gamma_i(k)}
\end{equation}

So, Equation (10), (11) and (12) are the main steps for M-Step.

Following Equation (6), (10), (11) and (12), we can iterate through E-Step and M-Step until reaching some stop criteria.

\end{description}

\newpage
\section*{\Large II. Code-from-Scratch}

\subsection*{\large 1. Methods}

\subsection*{\large 2. Results}

\newpage
\section*{\Large III. TensorFlow}

\subsection*{\large 1. Methods}

\subsection*{\large 2. Results}


%\begin{equation}
%\begin{split}
%	\frac{dE}{dw_j} & = \frac{d(\sum_i{((t_i - y_i)^2)})}{dw_j} \\
%					& = \sum_i{\frac{d(t_i - g(w' x_i + b))^2}{dw_j}}
%\end{split}
%\end{equation}


%\begin{figure}[H]
%\centering
%\includegraphics[width=1.0\textwidth]{./images/detector}
%\caption{\label{fig:detector_system} Single detector system}
%\end{figure}


\clearpage

%\bibliographystyle{plain}
%\bibliographystyle{unsrt}
%\bibliography{reference.bib}

\end{document}

