% Template for ICASSP-2016 paper; to be used with:
%          spconf.sty  - ICASSP/ICIP LaTeX style file, and
%          IEEEbib.bst - IEEE bibliography style file.
% --------------------------------------------------------------------------
\documentclass{article}
\usepackage{spconf,amsmath,graphicx}

% Example definitions.
% --------------------
\def\x{{\mathbf x}}
\def\L{{\cal L}}

% Title.
% ------
\title{Comparison of Unsupervised Pre-training Methods}
%
% Single address.
% ---------------
%\name{Author(s) Name(s)\thanks{Thanks to XYZ agency for funding.}}
%\address{Author Affiliation(s)}

\name{Jifu Zhao}
\address{University of Illinois at Urbana-Champaign\\
		 Department of Nuclear, Plasma, and Radiological Engineering\\
		 Urbana, Illinois 61801, USA}

% For example:
% ------------
%\address{School\\
%	Department\\
%	Address}
%
% Two addresses (uncomment and modify for two-address case).
% ----------------------------------------------------------
%\twoauthors
%  {A. Author-one, B. Author-two\sthanks{Thanks to XYZ agency for funding.}}
%	{School A-B\\
%	Department A-B\\
%	Address A-B}
%  {C. Author-three, D. Author-four\sthanks{The fourth author performed the work
%	while at ...}}
%	{School C-D\\
%	Department C-D\\
%	Address C-D}
%
\begin{document}
%\ninept
%
\maketitle
%
\begin{abstract}
Unsupervised 
\end{abstract}
%
\begin{keywords}
Unsupervised pre-training, PCA, kernel PCA, Auto-encoder
\end{keywords}
%
\section{Introduction}
\label{sec:intro}

% -------------------------------------------------------------------------
\begin{figure}[htb]

\begin{minipage}[b]{1.0\linewidth}
  \centering
  \centerline{\includegraphics[width=8.5cm]{./figures/image1.pdf}}
%  \vspace{2.0cm}
  \centerline{(a) Result 1}\medskip
\end{minipage}
%
\begin{minipage}[b]{.48\linewidth}
  \centering
  \centerline{\includegraphics[width=4.0cm]{./figures/image3.pdf}}
%  \vspace{1.5cm}
  \centerline{(b) Results 3}\medskip
\end{minipage}
\hfill
\begin{minipage}[b]{0.48\linewidth}
  \centering
  \centerline{\includegraphics[width=4.0cm]{./figures/image4.pdf}}
%  \vspace{1.5cm}
  \centerline{(c) Result 4}\medskip
\end{minipage}
%
\caption{Example of placing a figure with experimental results.}
\label{fig:res}
%
\end{figure}


% To start a new column (but not a new page) and help balance the last-page
% column length use \vfill\pagebreak.
% -------------------------------------------------------------------------
%\vfill
%\pagebreak


\section{RELATION TO PRIOR WORK}
\label{sec:prior}

You should differentiate what is new and how your work expands on
or takes a different path from the prior studies. An example might
read something to the effect: "The work presented here has focused
on the formulation of the ABC algorithm, which takes advantage of
non-uniform time-frequency domain analysis of data. The work by
Smith and Cohen \cite{Lamp86} considers only fixed time-domain analysis and
the work by Jones et al \cite{C2} takes a different approach based on
fixed frequency partitioning. While the present study is related
to recent approaches in time-frequency analysis [3-5], it capitalizes
on a new feature space, which was not considered in these earlier
studies."

\vfill\pagebreak

% -------------------------------------------------------------------------
\bibliographystyle{IEEEbib}
\bibliography{refs}

\end{document}
