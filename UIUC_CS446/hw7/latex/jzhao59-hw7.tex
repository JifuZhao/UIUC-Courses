\documentclass[12pt,letterpaper]{article}

\newenvironment{proof}{\noindent{\bf Proof:}}{\qed\bigskip}

\newtheorem{theorem}{Theorem}
\newtheorem{corollary}{Corollary}
\newtheorem{lemma}{Lemma} 
\newtheorem{claim}{Claim}
\newtheorem{fact}{Fact}
\newtheorem{definition}{Definition}
\newtheorem{assumption}{Assumption}
\newtheorem{observation}{Observation}
\newtheorem{example}{Example}
\newcommand{\qed}{\rule{7pt}{7pt}}

\newcommand{\assignment}[4]{
\thispagestyle{plain} 
\newpage
\setcounter{page}{1}
\noindent
\begin{center}
\framebox{ \vbox{ \hbox to 6.28in
{\bf CS446: Machine Learning \hfill #1}
\vspace{4mm}
\hbox to 6.28in
{\hspace{2.5in}\large\mbox{Problem Set #2}}
\vspace{4mm}
\hbox to 6.28in
{{\it Handed Out: #3 \hfill Due: #4}}
}}
\end{center}
}

\newcommand{\solution}[4]{
\thispagestyle{plain} 
\newpage
\setcounter{page}{1}
\noindent
\begin{center}
\framebox{ \vbox{ \hbox to 6.28in
{\bf CS446: Machine Learning \hfill #4}
\vspace{4mm}
\hbox to 6.28in
{\hspace{2.5in}\large\mbox{Problem Set #3}}
\vspace{4mm}
\hbox to 6.28in
{#1 \hfill {\it Handed In: #2}}
}}
\end{center}
\markright{#1}
}

\newenvironment{algorithm}
{\begin{center}
\begin{tabular}{|l|}
\hline
\begin{minipage}{1in}
\begin{tabbing}
\quad\=\qquad\=\qquad\=\qquad\=\qquad\=\qquad\=\qquad\=\kill}
{\end{tabbing}
\end{minipage} \\
\hline
\end{tabular}
\end{center}}

\def\Comment#1{\textsf{\textsl{$\langle\!\langle$#1\/$\rangle\!\rangle$}}}



\usepackage{graphicx,amssymb,amsmath, listings}
\lstset{language = Matlab}
\lstset{breaklines}
\lstset{extendedchars=false}

\oddsidemargin 0in
\evensidemargin 0in
\textwidth 6.5in
\topmargin -0.5in
\textheight 9.0in
\begin{document}

\solution{Jifu Zhao}{11/21/2015}{7}{Fall 2015}
% Fill in the above, for example, as follows:
%\solution{Jifu Zhao}{\today}{1}{Fall 2012}

\pagestyle{myheadings}  % Leave this command alone

\begin{enumerate}
\item {\bf Answer to problem 1}
\begin{enumerate}
\item[\bf (a). ]
According to the definition, we can get that:
$$P(w_j, d_i) = P(d_i)P(w_j|d_i) = P(d_i)\sum_k P(w_j|c_k)P(c_k|d_i) $$

$$\bf P(w_j, d_i) = P(d_i)\sum_k P(w_j|c_k)P(c_k|d_i)$$\\

\item[\bf (b). ]

$$P(c_k| w_j, \: d_i) = \frac{P(c_k, w_j, d_i)}{P(w_j, d_i)}$$

According to (a), $P(w_j, d_i) = P(d_i)\sum_k P(w_j|c_k)P(c_k|d_i)$\\

And, we also have:
$$P(c_k, w_j, d_i) = P(w_j|c_k)P(c_k) = P(w_j|c_k)P(c_k|d_i)P(d_i)$$

So, finally:

$$\bf P(c_k| w_j, \: d_i) = \frac{P(w_j|c_k)P(c_k|d_i)}{\sum_k P(w_j|c_k)P(c_k|d_i)}$$\\

\item[\bf (c). ]

Likelihood should be:
$$L = \prod_i \prod_j P(w_j, d_i)^{n(d_i, w_j)}$$
$$LL = \sum_i \sum_j n(d_i, w_j)log P(d_i, w_j)$$
Since that:
$$P(d_i, w_j) = P(d_i)\sum_k P(w_j|c_k)P(c_k|d_i)$$
$$LL = \sum_i \sum_j n(d_i, w_j)log[P(d_i)\sum_k P(w_j|c_k)P(c_k|d_i)]$$

Suppose that k can be 1 and 2, so:
$${\bf E[LL]} = \sum_i \sum_j n(d_i, w_j)\{ P(c_1|w_j, d_i)log[P(d_i) P(w_j|c_1)P(c_1|d_i)] $$
$$ + P(c_2|w_j, d_i)log[P(d_i) P(w_j|c_2)P(c_2|d_i)] \}$$\\

\item[\bf (d). ]

Given the value to be maximized:
$${\bf E[LL]} = \sum_i \sum_j n(d_i, w_j)\{ P(c_1|w_j, d_i)log[P(d_i) P(w_j|c_1)P(c_1|d_i)] $$
$$ + P(c_2|w_j, d_i)log[P(d_i) P(w_j|c_2)P(c_2|d_i)] \}$$\\

Also notice that:
$$\sum_i P(d_i) = 1$$
$$\sum_k P(c_k|d_i) = 1$$
$$\sum_j P(w_j|c_k) = 1$$

Solve these equations, solve for $\bf P(w_j|c_k)$ and $\bf P(c_k|d_i)$,  we can find that in order to maximize ${\bf E[LL]}$, we should have that:\\

$${\bf P(w_j|c_k)} = \frac{\sum_{i=1}^M n(d_i, w_j)P(c_k|d_i, w_j)}{\sum_{i=1}^M\sum_{l=1}^V n(d_i, w_l)P(c_k|d_i, w_l)}$$\\

$${\bf P(c_k|d_i)} = \frac{\sum_{j=1}^V n(d_i, w_j)P(c_k|d_i, w_j)}
{\sum_{j=1}^V n(d_i, w_j)}$$\\

Finally, estimate $P(d_i)$ to be $P(d_i) = 1/M$\\

\item[\bf (e). ]

From (d), we can know the expression for $\bf P(w_j|c_k)$ , $\bf P(c_k|d_i)$ and 
$\bf P(d_i)$. They are the corresponding update rule. \\

1, For $\bf P(w_j|c_k)$, the value of:
$$\frac{\sum_{i=1}^M n(d_i, w_j)P(c_k|d_i, w_j)}{\sum_{i=1}^M\sum_{l=1}^V n(d_i, w_l)P(c_k|d_i, w_l)}$$
 means that we first iterate all documents (i = 1 to M), calculate how many times that $w_j$ has appeared with category of $c_k$, then, this value is divided by the total number of documents that has category $c_k$.\\

2, For $\bf P(c_k|d_i)$, the value of:
$$\frac{\sum_{j=1}^V n(d_i, w_j)P(c_k|d_i, w_j)}
{\sum_{j=1}^V n(d_i, w_j)}$$

means that first, when given the particular document $d_i$, we calculate how many times the category $c_k$ appeared, and then this value is divided by the total number of words in the document $d_i$.\\

3, For $\bf P(d_i)$, we only need to randomly choose one document from all documents without any more information, so its value is: $P(d_i) = 1/M$.\\

 
\item[\bf (f). ] The pseudocode is shown as below:\\

\rule{15cm}{.1pt}
\begin{enumerate}
\item[\bf 1, ] Choose the initial value for $P(w_j|c_k)$ , $P(c_k|d_i)$ and 
$P(d_i)$. They can be random numbers, or you can guess them based on experience.\\

\item[\bf 2, ] Do iteration:
\begin{enumerate}
\item[I), ] According to the expression of $E(LL)$ from part(a), (b), (c), calculate the current log likelihood.
\item[II), ] According to part(d), calculate new values for $P(w_j|c_k)$ , $P(c_k|d_i)$ and $P(d_i)$.

\item[III), ] According to the new value for $P(w_j|c_k)$ , $P(c_k|d_i)$ and $P(d_i)$, do the next iteration.

\item[IV), ] Stop until the expression converged.\\
\end{enumerate}

\item[\bf 3, ] Output the final conclusion.

\end{enumerate}

\rule{15cm}{.1pt}\\
\end{enumerate}


\item {\bf Answer to problem 2}

\begin{enumerate}
\item[\bf (a). ]

In this question, it means that no matter we choose which one as the root node, finally, the directed tree we finally built is equivalent.\\

So, no matter which one $y$ as the root node, the probability 
$$P(y, x_1, x_2, ......) = P(y)\prod_i (x_i| Parents(x_i))$$
should be the same for all $y$.\\

\item[\bf (b). ]

The mutual information between x and y, is:
$$I(x, y) = \sum_{x, y} P(x, y)\frac{P(x, y)}{P(x)P(y)}$$
In this question, we can get that:
$$I(x_i, x_j) = \sum_{x_i, x_j} Px_i, x_j)\frac{P(x_i, x_j)}{P(x_i)P(x_j)}$$

So, it is easy to derive that: $I(x_i, x_j) = I(x_j, x_i)$, so, the weight for edge $(x_i, x_j)$ and $(x_j, x_i)$. \\

This means that, no matter which node in T is chosen as the root for the "direction" stage, since the weight for edge $(x_i, x_j)$ and $(x_j, x_i)$ are the same, so the resulting joint probability distribution is the same, so, the resulting directed trees are all equivalent.\\

\end{enumerate}

\end{enumerate}

\end{document}

